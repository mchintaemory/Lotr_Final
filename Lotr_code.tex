% Options for packages loaded elsewhere
\PassOptionsToPackage{unicode}{hyperref}
\PassOptionsToPackage{hyphens}{url}
%
\documentclass[
]{article}
\usepackage{amsmath,amssymb}
\usepackage{iftex}
\ifPDFTeX
  \usepackage[T1]{fontenc}
  \usepackage[utf8]{inputenc}
  \usepackage{textcomp} % provide euro and other symbols
\else % if luatex or xetex
  \usepackage{unicode-math} % this also loads fontspec
  \defaultfontfeatures{Scale=MatchLowercase}
  \defaultfontfeatures[\rmfamily]{Ligatures=TeX,Scale=1}
\fi
\usepackage{lmodern}
\ifPDFTeX\else
  % xetex/luatex font selection
\fi
% Use upquote if available, for straight quotes in verbatim environments
\IfFileExists{upquote.sty}{\usepackage{upquote}}{}
\IfFileExists{microtype.sty}{% use microtype if available
  \usepackage[]{microtype}
  \UseMicrotypeSet[protrusion]{basicmath} % disable protrusion for tt fonts
}{}
\makeatletter
\@ifundefined{KOMAClassName}{% if non-KOMA class
  \IfFileExists{parskip.sty}{%
    \usepackage{parskip}
  }{% else
    \setlength{\parindent}{0pt}
    \setlength{\parskip}{6pt plus 2pt minus 1pt}}
}{% if KOMA class
  \KOMAoptions{parskip=half}}
\makeatother
\usepackage{xcolor}
\usepackage[margin=1in]{geometry}
\usepackage{color}
\usepackage{fancyvrb}
\newcommand{\VerbBar}{|}
\newcommand{\VERB}{\Verb[commandchars=\\\{\}]}
\DefineVerbatimEnvironment{Highlighting}{Verbatim}{commandchars=\\\{\}}
% Add ',fontsize=\small' for more characters per line
\usepackage{framed}
\definecolor{shadecolor}{RGB}{248,248,248}
\newenvironment{Shaded}{\begin{snugshade}}{\end{snugshade}}
\newcommand{\AlertTok}[1]{\textcolor[rgb]{0.94,0.16,0.16}{#1}}
\newcommand{\AnnotationTok}[1]{\textcolor[rgb]{0.56,0.35,0.01}{\textbf{\textit{#1}}}}
\newcommand{\AttributeTok}[1]{\textcolor[rgb]{0.13,0.29,0.53}{#1}}
\newcommand{\BaseNTok}[1]{\textcolor[rgb]{0.00,0.00,0.81}{#1}}
\newcommand{\BuiltInTok}[1]{#1}
\newcommand{\CharTok}[1]{\textcolor[rgb]{0.31,0.60,0.02}{#1}}
\newcommand{\CommentTok}[1]{\textcolor[rgb]{0.56,0.35,0.01}{\textit{#1}}}
\newcommand{\CommentVarTok}[1]{\textcolor[rgb]{0.56,0.35,0.01}{\textbf{\textit{#1}}}}
\newcommand{\ConstantTok}[1]{\textcolor[rgb]{0.56,0.35,0.01}{#1}}
\newcommand{\ControlFlowTok}[1]{\textcolor[rgb]{0.13,0.29,0.53}{\textbf{#1}}}
\newcommand{\DataTypeTok}[1]{\textcolor[rgb]{0.13,0.29,0.53}{#1}}
\newcommand{\DecValTok}[1]{\textcolor[rgb]{0.00,0.00,0.81}{#1}}
\newcommand{\DocumentationTok}[1]{\textcolor[rgb]{0.56,0.35,0.01}{\textbf{\textit{#1}}}}
\newcommand{\ErrorTok}[1]{\textcolor[rgb]{0.64,0.00,0.00}{\textbf{#1}}}
\newcommand{\ExtensionTok}[1]{#1}
\newcommand{\FloatTok}[1]{\textcolor[rgb]{0.00,0.00,0.81}{#1}}
\newcommand{\FunctionTok}[1]{\textcolor[rgb]{0.13,0.29,0.53}{\textbf{#1}}}
\newcommand{\ImportTok}[1]{#1}
\newcommand{\InformationTok}[1]{\textcolor[rgb]{0.56,0.35,0.01}{\textbf{\textit{#1}}}}
\newcommand{\KeywordTok}[1]{\textcolor[rgb]{0.13,0.29,0.53}{\textbf{#1}}}
\newcommand{\NormalTok}[1]{#1}
\newcommand{\OperatorTok}[1]{\textcolor[rgb]{0.81,0.36,0.00}{\textbf{#1}}}
\newcommand{\OtherTok}[1]{\textcolor[rgb]{0.56,0.35,0.01}{#1}}
\newcommand{\PreprocessorTok}[1]{\textcolor[rgb]{0.56,0.35,0.01}{\textit{#1}}}
\newcommand{\RegionMarkerTok}[1]{#1}
\newcommand{\SpecialCharTok}[1]{\textcolor[rgb]{0.81,0.36,0.00}{\textbf{#1}}}
\newcommand{\SpecialStringTok}[1]{\textcolor[rgb]{0.31,0.60,0.02}{#1}}
\newcommand{\StringTok}[1]{\textcolor[rgb]{0.31,0.60,0.02}{#1}}
\newcommand{\VariableTok}[1]{\textcolor[rgb]{0.00,0.00,0.00}{#1}}
\newcommand{\VerbatimStringTok}[1]{\textcolor[rgb]{0.31,0.60,0.02}{#1}}
\newcommand{\WarningTok}[1]{\textcolor[rgb]{0.56,0.35,0.01}{\textbf{\textit{#1}}}}
\usepackage{graphicx}
\makeatletter
\newsavebox\pandoc@box
\newcommand*\pandocbounded[1]{% scales image to fit in text height/width
  \sbox\pandoc@box{#1}%
  \Gscale@div\@tempa{\textheight}{\dimexpr\ht\pandoc@box+\dp\pandoc@box\relax}%
  \Gscale@div\@tempb{\linewidth}{\wd\pandoc@box}%
  \ifdim\@tempb\p@<\@tempa\p@\let\@tempa\@tempb\fi% select the smaller of both
  \ifdim\@tempa\p@<\p@\scalebox{\@tempa}{\usebox\pandoc@box}%
  \else\usebox{\pandoc@box}%
  \fi%
}
% Set default figure placement to htbp
\def\fps@figure{htbp}
\makeatother
\setlength{\emergencystretch}{3em} % prevent overfull lines
\providecommand{\tightlist}{%
  \setlength{\itemsep}{0pt}\setlength{\parskip}{0pt}}
\setcounter{secnumdepth}{-\maxdimen} % remove section numbering
\usepackage{booktabs}
\usepackage{caption}
\usepackage{longtable}
\usepackage{colortbl}
\usepackage{array}
\usepackage{anyfontsize}
\usepackage{multirow}
\usepackage{bookmark}
\IfFileExists{xurl.sty}{\usepackage{xurl}}{} % add URL line breaks if available
\urlstyle{same}
\hypersetup{
  pdftitle={LOTR\_analysis},
  pdfauthor={Mirra Chinta},
  hidelinks,
  pdfcreator={LaTeX via pandoc}}

\title{LOTR\_analysis}
\author{Mirra Chinta}
\date{2025-10-01}

\begin{document}
\maketitle

\section{\texorpdfstring{\textbf{Introduction}}{Introduction}}\label{introduction}

The ``Lord of the Rings'' characters dataset from Kaggle contains
character descriptions from Peter Jackson's Lord of the Rings trilogy.
The following analysis will look at the number of characters within the
Lord of the Rings' fictional races and gender breakdown across
characters. The dataset contains 909 characters across various time
periods within the Lord of the Rings universe.

\section{\texorpdfstring{\textbf{Loading
data}}{Loading data}}\label{loading-data}

\begin{Shaded}
\begin{Highlighting}[]
\NormalTok{lotr }\OtherTok{\textless{}{-}} \FunctionTok{read.csv}\NormalTok{(}\StringTok{"lotr\_characters.csv"}\NormalTok{)}
\end{Highlighting}
\end{Shaded}

\section{\texorpdfstring{\textbf{Table}}{Table}}\label{table}

\begin{table}[t]
\caption{\label{tab:unnamed-chunk-3}\textbf{Gender distribution by Race}} 
\fontsize{12.0pt}{14.0pt}\selectfont
\begin{tabular*}{\linewidth}{@{\extracolsep{\fill}}lccccccccccccccccccccccccccccccccccccccccccccccccc}
\toprule
 &  & \multicolumn{48}{c}{\textbf{Race}} \\ 
\cmidrule(lr){3-50}
\textbf{Characteristic} & \textbf{Overall}  N = 911\textsuperscript{\textit{1}} & ****  N = 140\textsuperscript{\textit{1}} & \textbf{Ainur}  N = 24\textsuperscript{\textit{1}} & \textbf{Ainur,Maiar}  N = 1\textsuperscript{\textit{1}} & \textbf{Balrog}  N = 1\textsuperscript{\textit{1}} & \textbf{Black Uruk}  N = 1\textsuperscript{\textit{1}} & \textbf{Dragon}  N = 1\textsuperscript{\textit{1}} & \textbf{Dragons}  N = 5\textsuperscript{\textit{1}} & \textbf{Drúedain}  N = 1\textsuperscript{\textit{1}} & \textbf{Dwarf}  N = 1\textsuperscript{\textit{1}} & \textbf{Dwarven}  N = 1\textsuperscript{\textit{1}} & \textbf{Dwarves}  N = 42\textsuperscript{\textit{1}} & \textbf{Eagle}  N = 1\textsuperscript{\textit{1}} & \textbf{Eagles}  N = 1\textsuperscript{\textit{1}} & \textbf{Elf}  N = 3\textsuperscript{\textit{1}} & \textbf{Elves}  N = 103\textsuperscript{\textit{1}} & \textbf{Elves,Maiar}  N = 1\textsuperscript{\textit{1}} & \textbf{Elves,Noldor}  N = 1\textsuperscript{\textit{1}} & \textbf{Ents}  N = 2\textsuperscript{\textit{1}} & \textbf{Ents,Onodrim}  N = 1\textsuperscript{\textit{1}} & \textbf{Goblin,Orc}  N = 1\textsuperscript{\textit{1}} & \textbf{God}  N = 1\textsuperscript{\textit{1}} & \textbf{Great Eagles}  N = 2\textsuperscript{\textit{1}} & \textbf{Great Spiders}  N = 2\textsuperscript{\textit{1}} & \textbf{Half-elven}  N = 7\textsuperscript{\textit{1}} & \textbf{Half-elven,Men}  N = 1\textsuperscript{\textit{1}} & \textbf{Hobbit}  N = 16\textsuperscript{\textit{1}} & \textbf{Hobbits}  N = 126\textsuperscript{\textit{1}} & \textbf{Horse}  N = 1\textsuperscript{\textit{1}} & \textbf{Maiar}  N = 4\textsuperscript{\textit{1}} & \textbf{Maiar,Balrog}  N = 1\textsuperscript{\textit{1}} & \textbf{Maiar,Balrogs}  N = 1\textsuperscript{\textit{1}} & \textbf{Men}  N = 388\textsuperscript{\textit{1}} & \textbf{Men,Rohirrim}  N = 1\textsuperscript{\textit{1}} & \textbf{Men,Skin-changer}  N = 1\textsuperscript{\textit{1}} & \textbf{Men,Undead}  N = 1\textsuperscript{\textit{1}} & \textbf{Men,Wraith}  N = 3\textsuperscript{\textit{1}} & \textbf{Orc}  N = 4\textsuperscript{\textit{1}} & \textbf{Orc,Goblin}  N = 1\textsuperscript{\textit{1}} & \textbf{Orcs}  N = 6\textsuperscript{\textit{1}} & \textbf{Raven}  N = 1\textsuperscript{\textit{1}} & \textbf{Skin-changer}  N = 1\textsuperscript{\textit{1}} & \textbf{Stone-trolls}  N = 1\textsuperscript{\textit{1}} & \textbf{Uruk-hai}  N = 3\textsuperscript{\textit{1}} & \textbf{Uruk-hai,Orc}  N = 1\textsuperscript{\textit{1}} & \textbf{Urulóki}  N = 1\textsuperscript{\textit{1}} & \textbf{Vampire}  N = 1\textsuperscript{\textit{1}} & \textbf{Werewolves}  N = 2\textsuperscript{\textit{1}} & \textbf{Wolfhound}  N = 1\textsuperscript{\textit{1}} \\ 
\midrule\addlinespace[2.5pt]
Gender &  &  &  &  &  &  &  &  &  &  &  &  &  &  &  &  &  &  &  &  &  &  &  &  &  &  &  &  &  &  &  &  &  &  &  &  &  &  &  &  &  &  &  &  &  &  &  &  &  \\ 
    Male & 768 (100\%) & 3 (100\%) & 24 (100\%) & 1 (100\%) & 1 (100\%) & 1 (100\%) & 1 (100\%) & 5 (100\%) & 1 (100\%) & 1 (100\%) & 1 (100\%) & 42 (100\%) & 1 (100\%) & 1 (100\%) & 3 (100\%) & 100 (100\%) & 1 (100\%) & 1 (100\%) & 2 (100\%) & 1 (100\%) & 1 (100\%) & 1 (100\%) & 2 (100\%) & 2 (100\%) & 7 (100\%) & 1 (100\%) & 16 (100\%) & 126 (100\%) & 0 (NA\%) & 4 (100\%) & 1 (100\%) & 1 (100\%) & 386 (100\%) & 1 (100\%) & 1 (100\%) & 1 (100\%) & 3 (100\%) & 4 (100\%) & 1 (100\%) & 6 (100\%) & 1 (100\%) & 1 (100\%) & 1 (100\%) & 3 (100\%) & 1 (100\%) & 1 (100\%) & 1 (100\%) & 2 (100\%) & 1 (100\%) \\ 
\bottomrule
\end{tabular*}
\begin{minipage}{\linewidth}
\textsuperscript{\textit{1}}n (\%)\\
\end{minipage}
\end{table}

\section{\texorpdfstring{\textbf{Table
Description}}{Table Description}}\label{table-description}

This table contains the counts of characters for each race within the
Lord of Rings Universe, as well as a the number of characters of each
gender under each race with the percentage distribution. The races with
the highest character counts include the Ainur, Elves, Hobbits, and Men.
All races with LOTR characters are predominantly male as well.

\section{\texorpdfstring{\textbf{Figure}}{Figure}}\label{figure}

\pandocbounded{\includegraphics[keepaspectratio]{Lotr_code_files/figure-latex/unnamed-chunk-4-1.pdf}}

\section{\texorpdfstring{\textbf{Figure
Description}}{Figure Description}}\label{figure-description}

This histogram shows the discribution of race across Lord of The Rings
characters. Across the horizontal axis, are the fictional races listed
and across the y axis are counts.

\end{document}
